\section{Results}
\label{sec:results}

\subsection{Adenylate Kinase (\AK)}
\label{subsec:ak_results}

%To first validate the effectiveness of \Model, we employed 
\AK is a well-studied monomeric enzyme known for its large-scale conformational transitions. 
%, catalyzing the reversible phosphoryl transfer reaction (ATP + AMP $\leftrightarrow$ 2ADP) that regulates cellular energy homeostasis. 
\AK is composed of three relatively rigid domains: the central CORE domain (residues 1–29, 68–117, and 161–214), the AMP-binding domain (AMPbd; residues 30–67), and the lid-like ATP-binding domain (LID; residues 118–167). 
Experimental and computatinoal studies have suggested that upon ligand binding, the enzyme undergoes a transition from an inactive open conformation to an active closed conformation (see \Cref{fig:overview}~(c)). 
Here, we evaluated how accurately \Model predicts 3D conformations of \AK from AFM images.

%\paragraph{CNN training.}

First, we checked the quality of the conformations generated by \AFiii for training data of the g-CNN. The candidate conformations were generated with \AFiii  (by steering \AFiii with CV values on a grid as described in \Cref{subsubsec:afm-prep}). 
%are projected them to the inter-domain distance space (\Cref{fig:ak_cnn_training}~(b)).
%Results are shown in \cref{fig:ak_cnn_training}~(b). 
Here, we used the inter-domain distances of all possible combinations of domain pairs (LID--CORE, CORE--AMPbd, and AMPbd--LID) as CVs. 
% is equivalent to $\R^3$.
\Cref{fig:ak_cnn_training}~(b) shows the generated conformations projected to the LID--CORE, and CORE--AMPbd distances (colored by the AMPbd--LID distance). 
Also, \Cref{fig:ak_cnn_training}~(a) shows the projection of the conformations sampled by a 450 ns MD simulation (\Cref{fig:ak_cnn_training}~(a)), which covers both closed (PDB ID: 1AKE) and open (PDB ID: 4AKE) crystal structures.
Comparison of the two projections shows that the generated conformations adequately covered a broad conformational space of \AK.
%\Cref{fig:ak_cnn_training}~(a) is that of representive conformations in a 450 ns MD trajectory. 
%In this simulation, multiple transitions between the open and closed conformations were observed, suggesting that a wide conformational space was explored. 
%Importantly, the sampling broadly covered the conformational space that was explored by MD. 
%even though we used no prior knowledge about MD simulations.

%\paragraph{Steering ability.}
Next, we investigated how accurately \AFiii steering follows the specified CVs (inter-domain distances).
Since the CV values in a grid used for generating the candidate conformations include unphysical values, which may not be physically achievable, we evaluated the accuracy of steering by using the CV values of the crystal structures. % and the conformations sampled by the MD simulation. 
%\AFiii generated an open conformation when we did not impose any restraints. 
%As an initial test of \Model's ability to steer \AFiii's sampling trajectory, we generated pseudo-AFM images from the closed (PDB ID: 1AKE) and open (PDB ID: 4AKE) crystal structures, and reconstructed 3D conformations from each image. 
\Cref{fig:overview}~(b) and (c) show the generated structues by applying the CV values of the closed and open crystal structures.  
%The results are shown in \Cref{fig:overview}~(b), (c).
The RMSD between the ground-truth and reconstructed conformations was 0.216 nm for the closed form and 0.176 nm for the open form, respectively. 
Considering that the RMSD between the closed and open crystal structures is 0.715 nm, the steering was accurate enough to distinguish the two states. 
The pseudo-AFM images generated from predicted conformations after rigid-body fitting of the generated structures are displayed in \Cref{fig:overview}~(b). 
Using \Cref{eq:cc} to compute the c.c., we obtained 0.997 for the closed form and 0.998 for the open form.

\begin{figure}[H]
    \centering
    \includegraphics[width=0.7\linewidth]{images/figures-02.png}
    \caption{
        \textbf{Training data and prediction accuracy of the group-invariant CNN}.
        \textbf{(a)} Projection of a 450 ns MD simulation trajectory into the inter-domain distance space. 
        \textbf{(b)} Projection of the candidate conformations into the inter-domain distance space. 
        \textbf{(c)} A histogram of root squared error in the inter-domain distance space between the pre-trained CNN's outputs and ground-truth. 
        The CNN predicted inter-domain distance from pseudo-AFM images, which ware generated from randomly selected MD conformations. 
        The root mean squared error (RMSE) was 0.183 [nm].
        \textbf{(d)} A scatter plot of root squared error in the inter-domain distance space and ${\mathrm{C}\alpha}$-atom RMSDs. 
        The RMSD was computed between the conformation generated with \AFiii steering using the predicted CV values, and the ground-truth conformation.
        }
    \label{fig:ak_cnn_training}
\end{figure}

%\paragraph{Accuracy evaluation.}
Finally, we investigated the accuracy of the estimated conformatinos of our \Model framework. Here, we estimated CV values from pseudo-AFM images generated from representative conformations of the 450 ns MD simulation, and steered \AFiii using the estimated CV values to generated 3D conformations. The pseudo-AFM images were generated with the same settings as those in \Cref{tab:afm_settings}. 
%To investigate the prediction accuracy of \Model, we extracted representative conformations from the 450 ns MD simulation and generated pseudo-AFM images (settings identical to those in \cref{tab:afm_settings}).
%Using these images, we predicted 3D conformations with the \Model framework.
\Cref{fig:ak_cnn_training}~(c) shows the histogram of the root squared error (RMSE) in the CV space between the predicted values with the trained g-CNN and the ground-truth values.
% and \cref{fig:ak_accuracy}~(a)--(c).
%An informative observation is that we can analyze sources of error in \Model's 3D conformation reconstruction.
%From the histogram of the root square error between the pre-trained CNN predictions and the ground-truth inter-domain distances (\cref{fig:ak_cnn_training}~(c))
With the exceptions of some outlier errors greater than RMSE of 0.5 nm, the trained g-CNN achieves the average RMSE of 0.183 nm, which is enough resolution to capture intermediate conformational states. 

\Cref{fig:ak_cnn_training}~(d) shows the ${\mathrm{C}\alpha}$-atom RMSDs between the generated conformations with \AFiii steering and the ground-truth conformations.
Based on a linear regression, the CNN’s typical prediction error corresponds to an RMSD of 0.245 nm. 
However, \Cref{fig:ak_cnn_training}~(d) also shows cases where the RMSD remains relatively large even when the CNN predictions are accurate: specifically, points in the upper-left region exhibit small RMSE but large RMSD.
Close inspection of such cases reveals that this discrepancy arises when domains move along a shear direction: inter-domain distances remain similar, but other degrees of freedom differ, yielding a larger RMSD.

A second factor that degrades reconstruction accuracy is the quality of the images themselves.
\Cref{fig:ak_accuracy}~(a) shows the distribution of RMSD between the ground-truth and predicted conformations.
While most cases exhibit low RMSD, a subset shows high RMSD in \Cref{fig:ak_accuracy}~(a).
\Cref{fig:ak_accuracy}~(b) shows the distribution of the c.c. between the reference images and the pseudo-AFM images generated by rigid-body fitting of the predictions to the reference.
In contrast to RMSD, \Cref{fig:ak_accuracy}~(b) is skewed toward high c.c.
Indeed, \Cref{fig:ak_accuracy}~(c) indicates that even when RMSD is relatively high, c.c. can remain high (upper-right region), suggesting that some of the reference AFM image does not sufficiently determine inter-domain distances.
Taken together, despite the presence of systematic errors stemming from methodological limitations of \Model, the framework reliably distinguishes large-scale domain motions with high probability overall.

\begin{figure}[htbp]
    \centering
    \includegraphics[width=\linewidth]{images/figures-03.png}
    \caption{
        \textbf{Evaluation of prediction accuracy}. 
        \textbf{(a)} Histogram of RMSD between predicted and ground-truth structures (mean: 0.273 nm). 
        \textbf{(b)} Histogram of correlation coefficients (C.C.) between reference images and rigid-body fitted predicted images (mean: 0.997). 
        \textbf{(c)} Scatter plot of C.C. and RMSD, indicating a negative correlation between image similarity and structural accuracy.
    }
    \label{fig:ak_accuracy}
\end{figure}

\paragraph{Noise Robustness}
To assess how \Model performs under varying noise conditions, we evaluated its robustness to image noise.
From \cref{fig:noise_robustness}, we observed the following.
When we used noise-free AFM images as training data, the model accurately extracted features and inferred structures from noise-free inputs.
However, it lacked robustness to noisy inputs: inference accuracy degraded substantially when the reference images (inputs) were heavily contaminated by noise.
By contrast, using images with comparatively high noise levels for training introduced a trade-off between peak accuracy and robustness.
Specifically, even when the input noise matched the training noise, the prediction accuracy did not reach that of the noise-free case.
Nevertheless, unlike the noise-free setting, the model retained reasonable accuracy even when the input noise level differed from that of the training data.

In general, the noise level of the training data should match that of the target images.
Based on these results, the recommended practice depends on the noise level of the reference images.
If the reference images have high noise levels ($\gtrsim 0.3$ nm), one should accept somewhat lower reliability, but reasonable accuracy can still be expected even when the input noise level differs from the training data.
Conversely, if the reference images are low-noise, highly reliable predictions are achievable, but only when the input noise level is similar to that of the training data; images with markedly different noise levels should not be used as references.

\subsection{\flhac}
\label{subsec:flhac_results}

FlhA is the largest subunit of the flagellar protein export apparatus, and its C-terminal cytoplasmic domain, \flhac, is thought to assist secretion by facilitating the binding of soluble proteins to the transporter, thereby promoting their export.
Structurally, \flhac comprises four domains (\acdi, \acdii, \acdiii, and \acdiv) and a linker to the transmembrane domain (FlhA\textsubscript{TM}).
Previous studies using X-ray crystallography and mutational analysis indicate that hinge motions among these domains influence transport activity and motility \cite{saijo-hamano2010flha}.

\paragraph{Preprocessing}
We selected \acdi--\acdiv and \acdii--\acdiii as inter-domain distance features.
The candidate conformations and their MolProbity scores are shown in \cref{fig:flhac_selection}.

\paragraph{Comparison with Rigid-Body Fitting}
To evaluate the applicability of \Model to real AFM data, we selected three snapshots from HS-AFM movies of the \flhac monomer and predicted a conformation from each.
The frames consisted of one image resembling an open state, one resembling a closed state, and one distinct from both.
For each image, we predicted the conformation and reproduced AFM images by rigid-body fitting of the predicted conformation (see \cref{subsubsec:afm-evaluation}).
For comparison, rigid-body fitting was also performed using X-ray crystal structures obtained from the PDB.

\Cref{fig:flhac_results}~(a) shows the selected reference images (upper: open-like; middle: closed-like; lower: distinct).
\Cref{fig:flhac_results}~(b) shows the AFM images reproduced by rigid-body fitting (for each, top: \Model predictions; bottom: crystal structures).
In \cref{fig:flhac_results}~(c), the correlation coefficients (c.c.) across rotations are plotted.

For both the \Model predictions (\cref{fig:flhac_results}~(c), top) and the crystal structures (\cref{fig:flhac_results}~(c), bottom), the optimal pose obtained by rigid-body fitting (star symbol) generally captured the global maximum of c.c.
Furthermore, \cref{fig:flhac_results}~(d) presents histograms of c.c. over uniformly sampled $\mathrm{SO}(3)$ rotations.
In all cases, results using the conformations that \Model predicted (\cref{fig:flhac_results}~(d), top) show higher c.c. with the reference images than those using the crystal structures (\cref{fig:flhac_results}~(d), bottom).

Of particular note is the result for the final reference image (lower).
The reference image (\cref{fig:flhac_results}~(a), lower) exhibits a highly elongated shape, suggesting that conformations in which the four domains are aligned in a parallel would yield high correlations.
However, \Model did not produce such an extremely open conformation.
This limitation arises because the CNN performs interpolation but not extrapolation: the CNN's prediction is constrained by the candidate conformations, and it cannot output conformations far beyond them.
%Since we did not find extremely open and stable conformations during candidate generation, \Model did not predict such conformations.

To mitigate this issue, two approaches may be possible: (i) explore a broader range of candidate conformations and generate training data with weights according to thermodynamic stability; or (ii) replace \AFiii as the base model with an ensemble generative model that has a broader prior distribution.

\begin{figure}[htbp]
    \centering
    \includegraphics[width=0.9\linewidth]{images/figures-04.png}
    \caption{
        \textbf{Results for \flhac}.
        \textbf{(a)} Reference AFM images selected from real experimental data. 
        \textbf{(b)} Results of rigid-body fitting. 
        For each reference image, we reconstructed 3D conformations and database structures, applied rigid-body fitting, and plotted the resulting optimal poses along with their corresponding pseudo-AFM images. 
        For each row, the top figures show the results for \Model’s predictions, and the bottom figures show the results for the database structures. 
        \textbf{(c)} Visualization of the rigid-body fitting process. 
        For all sampled SO(3) rotations, the corresponding correlation coefficients (c.c.) are color-mapped: blue indicates lower c.c. (worse fits), while red indicates higher c.c. (better fits). 
        The star denotes the optimal pose. 
        \textbf{(d)} Histogram of c.c. values. 
        We uniformly sampled SO(3) rotations and computed the c.c. for each case. 
        The histogram shows the resulting distribution, and the top-right annotation reports both the maximum c.c. (i.e., at the optimal pose) and the mean c.c. value.
    }
    \label{fig:flhac_results}
\end{figure}
